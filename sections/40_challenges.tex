\section{Challenges}
\label{sec:challenges}

Of course, there are several challenges that need to be addressed when implementing such a dashboard. The most important challenges are: Availability of data, privacy concerns and effectiveness.

\subsection{Availability of Data}

When implementing such a dashboard, one of the key issues is the availability of data. Which kinds of data are available and from what sources is highly dependant on the institution. As discussed in Section \ref{sec:data}, in the case of the University of Bamberg, there are several sources of data that would need to be combined to create a comprehensive dashboard, and it is not clear that all data is already readily available.

Especially for the planning aspects, it is important to have data about all courses and degree programs available. Usually, degree program specifications are handed out as PDFs, which are not easily machine-readable. This data would need to be converted into a machine-readable format to be used in a dashboard. The system would also need to handle changes to degree programs, which might apply to every student (because some courses are no longer offered) or only to new students (because the degree program has been adapted).
Some universities also offer degree programs in cooperation with other universities, which would make the data gathering even more complex \cite{Dashboard-StudentProgress}.

\subsection{Privacy Concerns}
There are multiple ways in which privacy and data protection concerns could arise. For example, if the dashboard is used to suggest interventions in case of problems, it is important to ensure that the data is only accessible to the student and the responsible staff members.
In general, it is important to ensure that the data is only accessible to the people who need it, and that the data is not misused. Students should be allowed to decide who can see their data, and should be able to revoke access at any time. This is especially important when the data is used to suggest interventions, as this could be seen as an intrusion into the private life of the student.

Especially visualizations in which the student is compared to others are a sensitive topic. It is important that the data shown to the student does not allow for the identification of other students. For small courses, this might make it impossible to show certain visualizations, as other student would be easily identifiable.
Therefore, comparative visualizations need to make strike a balance between showing useful information and protecting the privacy of the students. This is especially important when students are compared to cohorts instead of the entire student body, which, as discussed previously, might improve the information that can be gained from the visualization but also decreases the anonymity set.

\subsection{Effectiveness}

The most important challenge is to ensure that the visualizations are actually effective. This means that the dashboard should help students to succeed in their academic career. One very important thing to consider is motivation of students.
If the visualizations are too negative, students might be tempted to give up. For example, if a student is shown that they are behind the average of their cohort, with a small chance of passing the exam, they might become demotivated and drop the class instead of making an effort and working harder.
On the other hand, if they know that they are ahead of the average, they might become complacent and work less, thus actually decreasing their chance of success.
Therefore, it is very important to conduct field test to ensure that the visualizations are not demotivating students.
