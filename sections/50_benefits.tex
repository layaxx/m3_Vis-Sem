\section{Benefits}
\label{sec:benefits}

The benefits of a student dashboard with such visualizations can be manifold.
The most important benefits are: Good overview for students, easier planning, early detection of problems, personalized feedback, and increased motivation.

When executed well, such a dashboard could reduce problems from the very beginning of the academic career, by allowing students to plan their degree program more effectively.
If and when problems arise, students could be supported by personalized feedback and suggestions for interventions, hopefully motivating them to participate and study more, thereby increasing increasing their chance to complete all courses. If a student does fail an exam, they could be assisted in planning their next steps. Depending on the type of course, this might mean choosing when to retake the exam, or choosing a different course that is more suitable for them.

There are benefits of such a dashboard that not only apply to students, but also to the entire university. Depending on how exactly the dashboard is implemented and data privacy concerns are mitigated, instructors and administrators could also benefit from the insights gained from the data as well as from higher graduation rates.

Another potential benefit that has not been discussed so far is the social aspect of studying. While this would certainly have even more privacy implications that would need to be addressed, it could be beneficial for students to not only see, how they perform compared to peers, but also to see how they could help each other.
For example, if a student is struggling with a course, they could be shown other students who have already passed the course and could be contacted for help.
In the pre-semester planning phase, it could also be interesting to take into account what courses other students are planning to take, to make it easier to find study groups, especially if those students have a similar profile or here have already been some common classes.
However, I am not aware of any research that has been done in this area, so this is purely speculative.
